\documentclass[sigconf]{acmart}

\usepackage{hyperref}

\usepackage{endfloat}
\renewcommand{\efloatseparator}{\mbox{}} % no new page between figures

\usepackage{booktabs} % For formal tables

\settopmatter{printacmref=false} % Removes citation information below abstract
\renewcommand\footnotetextcopyrightpermission[1]{} % removes footnote with conference information in first column
\pagestyle{plain} % removes running headers

\begin{document}
\title{Big Data in Financial Services}


\author{Huiyi Chen}
\orcid{hid101}
\affiliation{%
  \institution{Indiana University}
  \date{October 2017}
}
\email{huiychen@indiana.edu}



% The default list of authors is too long for headers}
\renewcommand{\shortauthors}{H. CHEN}


\begin{abstract}
This paper provides an insight of how big data has changed the world of financial services. It will identify some key challenges that businesses might face due to the blooming of big data \LaTeX\ It will also describe some applications of big data in financial services, how to approach to big data in finance in practice, and finally, how to get the right outcome.
\end{abstract}

\keywords{Big Data, Financial Services, Finance, Data}


\maketitle

\section{Introduction}

Big data has been around for a long time now. Many scientists and researchers have put in a lot of hours to investigate how big data can benefit human beings in different aspect. One of the most popular fields that people do research in is finance. That reason why it is so crucial to understand big data in finance is that finance is immersed in our daily life, and the cost of generating data and activities within these financial services has long been high. With the increased capability and reducing cost of advanced technology of big data, the potential of improvement in customer engagement and operating performance has increased, promoting businesses to invest more in the big data. The ability to access, analyze, and manage great volumes of data in the same time with evolving the Information Architecture has been crucial to financial services as they improve their business efficiency and performance.\cite{Stackowiak2015} Big data will be the biggest help these financial services could utilize. According to the article How Big Data Has Changed Finance by Trevir Nath, in the years of 2013 to 2015, 90 percent of all the data worldwide has been created as a result of the start of 2.5 quintillion bytes of data everyday.\cite{Nath2015} With that being said, how do we use the data in financial services?


\section{3 V's of Big Data}
The three V's are the fundamental to big data. They are volume, velocity, and variety. \cite{Nath2015} In the past few years, financial services have not been doing so great due to the increasing competition from both within the field and other higher rate of return investment, financial services are constantly seeking new methods to leverage technology to improve efficiency and profit. Using big data will definitely help financial services to develop a competitive advantage. \cite{Nath2015} Coming back to the three fundamentals to big data, velocity is the speed data must be store and analyzed at. According to New York Stock Exchange, it captured 1 terabyte of information everyday. With that being said, by 2016, there will be estimatedly 18.9 billion network connection, making roughly 2.5 connects per person in the world. Financial services can definitely differentiate themselves by focusing on quickly and efficiently processing trade from their competitors.\cite{Turner2013}

\section{Algorithmic Trading}
While back in the days, people make phone calls or come to the trading center to trade and to manage their banking accounts, technology has made this process a lot efficient for clients. Instead of using the traditional operational systems such as ATMs, call centers, or branches and brokerage units, they could now transfer money using the secure internet and trade stocks with one and another at the speed of a click. With the process becoming so convenient for clients, the financial services side need to speed up their processing time in order to keep up with the clients' needs. They can no longer rely on the traditional enterprise data from these old operational systems. They need to forecast more potential client information from sources such as industry data, trading data, news, and analyst reports from internal or competing banks. Big data shows its remarkable capability in terms of trading. With the growing capabilities of computers algorithmic trading becomes possible. The automated process enables many computer programs to execute financial trades at the speeds and frequencies that any human trader cannot. Within the mathematical models, algorithmic trading provides trades executed at the best possible prices and timely trade placement, and reduces manual errors due to behavioral factor\cite {Turner2013}



\section{Getting on track with big data}
It is granted that big data is powerful in the field of finance. However, it must be There are increasing number of big data applications in the education and standardized testing fields for learning these tests and predict performance and measure proficiency. One area that has been focusing on is knowledge discovery (KDD). It is an interdisciplinary area that focus on methodologies of identifying and retrieving meaningful patterns from large data set, for example, the College Board  testing data. KDD draws research in statistics, pattern recognition, machine learning and etc. to analyze testing data to make sure the tests are focusing on the important content but not uncommon knowledge that will potentially lower students' scores without proficiently indicating their abilities\cite {Daniel2015}. There is a variety of method we can use to conduct data mining research in education. Within these different approaches, different tasks can be done.

\subsection{Key Challenges Businesses face}
Almost all companies in the financial services have business intelligence tools and data warehouse for reporting and analyzing customer behavior to optimizing operations and anticipate clients' needs.

\subsection{Classification}
Classification is another approach in data mining research that helps to classify test data into predetermined categories. This approach is often referred to machine learning. These algorithms can help us learn form a large set of pre-classifled data, for example, reading, vocabulary, and math are three different categories that we can analyze in SAT \cite {Daniel2015}. To dig even deeper into each categories, we can even categorized different types of math questions such as algebra, calculus, and etc. 

\subsection{Association}
Association approach is also a kind of algorithm to extract the characteristics of some data within a set. These are driven by rule-based algorithms, which usually examine correlation between variables. This approach is to study the frequency associated with a data set \cite {Daniel2015}.

\subsection{Factor analysis}
One of the most common approach in education is Factor analysis. It is to find the different variables that could be grouped together, or split the set of variables into a set of factors \cite {Daniel2015}. ''Within the education field, factor analysis includes algorithms such as principal component analysis and exponential-family principal components analysis and used for dimensional reduction, including in pre-processing to reduce the potential for over-fitting and to determine meta-features. Factor analysis has also been used in learning environments to develop predictive models (Baker & Yacef, 2009; Minaei-Bidgoli, Kashy, Kortmeyer & Punch, 2003).''\cite {Daniel2015}


\section{Applications in education administration}
Not only big data can be used in analyzing standardized testing data, it can also be applied to many higher education administrations including admission processing, student performance monitoring in colleges and etc\cite {Picciano2012}. With the similar data mining methods, big data is going to make managing students' and staff's information more convenient with more significant and useful data. While in the past, we need to go to the registration office to register for classes, or the staff need to take a long time to schedule around classrooms, professors, times, and sections, we can now use something called course management or learning management system (CMS) \cite {Picciano2012}. This system gives us immediate report of student activities and course arrangements. It would also record and analyze students' homework and grades to determine what would need to be done to bring up students' grades or 
generate a class report on testing to compare with previous classes for the professor to make decisions on potential class materials. Nowadays, a CMS becomes critical for institutions to collect data to process important learning analytic application.

\section{Conclusions}
Competition in education will continue to be increasingly strong due to the growing population of students all over the world\cite {Selingo2017}. Both standardized tests and institutions are under pressure of responding to better the education systems with quality of learning programs and testing outcomes. More and more organizations and testing center are utilizing data to make better decisions about learning programs and questions distribution \cite {Selingo2017}. While higher education is largely rely on a political and economic situation of a country, big data will help these institutions and testing centers to maximize their ability to utilized existing data to generate the most information. Data-driven decision making is already in use to help institutions and testing centers to build strategies. Big data will continue to prove its work in education and standardized testing by making them more efficient and useful in the future. 




\bibliographystyle{ACM-Reference-Format}
\bibliography{report} 

\end{document}
